\documentclass[10pt, letterpaper]{article}
% \usepackage{graphicx}

\title{User manual for the package `dpch`}
\author{Sumeet Shirgure \thanks{University of Southern California}}
\date{September 2023}

\begin{document}
\maketitle

\section{Introduction}
This is a header only library, and is intended to be used as is.
There is no need to install headers to a specific location, but the Makefile contains a `make install` target to create a directory
called `dpch` in /usr/include where the header files can be stored.
Note that installing in /usr/include might require super user privileges.

To make sure your compiler finds the header files during usage, add a -I flag
followed by wherever you have stored the header library.
Also note that most systems have their compiler search inside /usr/include,
so if that's where the headers are situated the -I flag shouldn't be necessary.

Please refer to the makefile for more examples on compiler flags.
The compiler must be C++20 compliant for the templates in this library to work.

Please see the example.cc test driver code for more API details.

The `make` command will by default build tests.
The developer need not build them as they will always pass.

\section{Sample usage}

Please run the following commands to build the example binary,
and go through the example.cc file for a sample API usage.
\begin{verbatim}
  $ sudo make install
    ... Installs header to /usr/include ...
  $ c++ -std=c++20 example.cc -o example
  $ ./example
    All assertions passed.
  $ rm example # If no longer needed
\end{verbatim}
% $

\section{API}
The library provides two headers, one for the online hull that doesn't support
point deletion, and one for the dynamic hull that does support point deletions.
They share most of the API, except for an added function \verb|remove_point()|
in the dynamic hull.
Hence the C++ interface of just the dynamic hull is shown below :

\begin{verbatim}
  template <typename Field> class DynamicHull {
  public:
    void add_point(Point<Field> const&);
    /* remove_point not present in Online hull */
    bool remove_point(Point<Field> const&);
    std::optional< std::pair< Point<Field>, Point<Field> > >
      get_tangents (Point<Field> const&);
    std::pair< Point<Field>, Point<Field> >
      get_extremal_points(Point<Field> const&);
  private:
    // rest of the data structures and algorithms
  };
\end{verbatim}


\section{Manual}

Once the headers are located somewhere your compiler can find
(e.g. the /usr/include directory) the following include should work.

\verb|#include <dpch/util/Point.hh>|

Add the line \verb|using namespace dpch;| for convenience.

This header defines a 2D point class named \verb|Point<Field>| where 
`Field` is a template parameter.
For discrete fields like the integers, the hull computation becomes exact.
E.g if you want an integral point, define one as
\verb|auto point = Point<int>(3, 4);|

To create a dynamic planar hull data structure, use the following lines

\verb|#include <dpch/dynamic/DynamicHull.hh>|

\verb|DynamicHull<int> hull;|

To add a point to it, you can use the function

\verb|hull.add_point(point);|

Once you've added a few points like so, it's time to get the tangents from a
query point using the \verb|get_tangents()| function like so :

\verb|auto tangents = hull.get_tangents(query_point)|

This is a lot to unpack, so let's take it step by step.
If the \verb|query_point| was in the interior of the hull at this point,
then there are no tangents. To check this condition, do :

\verb|tangents == std::nullopt ? <inside> : <outside>|

If this condition fails, \verb|tangents.value()| will store a pair of points
where the line segments joining them to the query point will form the two
tangents. To retrieve the corresponding points on the hull, do :

\verb|tangents.value().first, tangents.value().second|

Now let's say your hull has some shape, and you want to find the farthest
set of points along the direction given by :

\verb|auto direction = Point<Field>(4, 3);|

You can find them quickly using the following line :

\verb|auto far_points = hull.get_extremal_points(direction);|

This returns a pair of points that you can get by doing

\verb|far_points.first| and \verb|far_points.second|

The line segment joining these two points is the section of the hull
that maximizes the dot product along the \verb|direction| vector.
If the two points are the same, which will usually be the case,
then the said line segment is a degenerate point on the hull.

Finally, to remove a point from the set of added points so far, do

\verb|hull.remove_point(point)|

Note that the point must be \emph{exactly} the same as the one inserted
some point earlier. \verb|dpch| doesn't consider proximity as equality.

\end{document}
